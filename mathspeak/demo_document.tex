# Introduction to Calculus

Calculus is the mathematical study of continuous change. It has two major branches: differential calculus and integral calculus.

## The Derivative

The derivative of a function $f(x)$ at a point $x$ is defined as:

$$f'(x) = \lim_{h \to 0} \frac{f(x+h) - f(x)}{h}$$

This represents the instantaneous rate of change of the function at that point.

### Example: Power Rule

For any function $f(x) = x^n$, the derivative is:

$$\frac{d}{dx} x^n = nx^{n-1}$$

Let's apply this to $f(x) = x^3$. We get $f'(x) = 3x^2$.

## The Integral

The integral is the inverse operation of the derivative. The definite integral of $f(x)$ from $a$ to $b$ is:

$$\int_a^b f(x) dx = F(b) - F(a)$$

where $F(x)$ is any antiderivative of $f(x)$.

### Fundamental Theorem of Calculus

The fundamental theorem states that:

$$\frac{d}{dx} \int_a^x f(t) dt = f(x)$$

This connects differentiation and integration in a profound way.

## Applications

Calculus has numerous applications in physics, engineering, and economics. For instance, in physics:

- Velocity is the derivative of position: $v = \frac{dx}{dt}$
- Acceleration is the derivative of velocity: $a = \frac{dv}{dt}$
- Work is the integral of force: $W = \int F dx$

### Example: Motion

If an object's position is given by $x(t) = t^2 + 3t + 1$, then:

- Velocity: $v(t) = \frac{dx}{dt} = 2t + 3$
- Acceleration: $a(t) = \frac{dv}{dt} = 2$

The object has constant acceleration!

## Advanced Topics

As we progress, we'll explore:

1. Partial derivatives: $\frac{\partial f}{\partial x}$
2. Multiple integrals: $\iint_R f(x,y) dA$
3. Vector calculus: $\nabla \times \vec{F}$
4. Differential equations: $\frac{dy}{dx} + P(x)y = Q(x)$

Each topic builds upon the fundamental concepts we've introduced here.